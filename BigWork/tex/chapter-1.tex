\documentclass[../document.tex]{subfiles}
\begin{document}
\section{我是一级标题}

{\heiti 这一节重点介绍小标题、符号、代码的使用。}就像这个样子,就限定了字体的使用范围,如果要强调,就能够像我这个样子。如果想使用脚注就使用后垢面这个\footnote{我是脚注的内容}, 我还想再脚注中添加公式\footnote{
比如放置一个条件概率公式,$P(X|Y)$
$$P(X|Y) = \frac{P(XY)}{P(Y)}$$
}

\subsection{我是二级标题}
这里能够写二级子目录的东西。
这里能够写二级子目录的东西。
这里能够写二级子目录的东西。
这里能够写二级子目录的东西。
这里能够写二级子目录的东西。

\subsection{我是二级标题}
\subsubsection{我是三级级标题}

我是段落内容。
我是段落内容。
我是段落内容。
我是段落内容。
我是段落内容。
我是段落内容。
我是段落内容。
我是段落内容。
我是段落内容。
我是段落内容。
我是段落内容。
我是段落内容。
我是段落内容。
我是段落内容。
我是段落内容。

\subsubsection{我是三级级标题}
我是段落内容。
我是段落内容。
我是段落内容。
我是段落内容。\\
我是段落内容。
我是段落内容。
我是段落内容。
我是段落内容。
我是段落内容。
我是段落内容。
我是段落内容。
我是段落内容。
我是段落内容。

我是段落内容。
我是段落内容。
我是段落内容。
我是段落内容。
我是段落内容。
我是段落内容。

\subsection{我是二级标题}
这里将介绍怎么编写公式:

我们能够简单的输入问本:
\begin{equation}
	P(XY) = P(Y|X) * P(X)
	\label{eq:乘法公式} 
\end{equation}
使用$equation$这个就是输入公式的环境,就能输入独占一行的公式,使用\$ 包括只能输入,与文本在一行,\$\$ 也能独占,但是不编号。

使用$ref$就能实现对公式的引用,将(\ref{eq:乘法公式})变换为条件概率公式,如下:
$$P(Y|X) = \frac{P(XY)}{P(X)}$$

这样就完成了公式的输入。其中frac输入公式的命令,在导航栏中就能找到所有的这样的命令,就能完成所有公式的输入了.如果问题还是蛮大就去看看别人的博客,进行公式入门就能够非常好的显示公式了。

\subsection{我是二级标题}

添加代码的时候就只需要按照下面的方式添加代码就好了:

\begin{lstlisting}
def viterbi(obs, states, start_p, trans_p, emit_p):
	V = [{}] #tabular
	path = {}
	for y in states: #init
		V[0][y] = start_p[y] * emit_p[y].get(obs[0],0)
		path[y] = [y]
	for t in range(1,len(obs)):
		V.append({})
		newpath = {}
		for y in states:
			(prob,state ) = max([(V[t-1][y0] * trans_p[y0].get(y,0) * emit_p[y].get(obs[t],0) ,y0) for y0 in states if V[t-1][y0]>0])
			V[t][y] =prob
			newpath[y] = path[state] + [y]
		path = newpath
	(prob, state) = max([(V[len(obs) - 1][y], y) for y in states])
	return (prob, path[state])
\end{lstlisting}
\end{document}